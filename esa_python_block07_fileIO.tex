\documentclass{beamer}

\usetheme{default}
\beamertemplatenavigationsymbolsempty

%http://mikedewar.wordpress.com/2009/02/25/latex-beamer-python-beauty/
\definecolor{fore}{RGB}{249,242,215}
%\definecolor{fore}{RGB}{51,51,51}
%\definecolor{back}{249,242,215}
\definecolor{back}{RGB}{51,51,51}
%\definecolor{back}{RGB}{249,242,215}
%{RGB}{51,51,51}
\definecolor{title}{RGB}{230,96,6}
%{255,0,90}
%http://meyerweb.com/eric/tools/color-blend/
%http://www.census.gov/population/international/data/worldpop/table_population.php
\setbeamercolor{titlelike}{fg=title}
\setbeamercolor{normal text}{fg=fore,bg=back}

\useinnertheme{rectangles}
\definecolor{UniBlue}{RGB}{34,139,34}
\setbeamercolor*{item}{fg=UniBlue}

\usepackage{colortbl}
\usepackage{caption}
\usepackage{listings,bera}
\usepackage{lipsum}
\newcommand\Fontvi{\fontsize{6}{7.2}\selectfont}
\newcommand\Fontvii{\fontsize{7}{7.5}\selectfont}
\newcommand\Fontviii{\fontsize{8}{7.8}\selectfont}
\newcommand\Fontix{\fontsize{9}{8.3}\selectfont}
\definecolor{keywords}{RGB}{255,102,0}
%{255,0,90}
\definecolor{comments}{RGB}{51,153,204}
%\definecolor{comments}{RGB}{83,121,170}
\setbeamertemplate{caption}[numbered]
%\useoutertheme{infolines}
\lstset{language=Python,
keywordstyle=\color{keywords},
commentstyle=\color{comments}\emph}
\setbeamertemplate{footline}{\scriptsize{\vspace*{0.3cm}\hspace*{0.3cm}\insertframenumber}} 

%\logo{epa-logo-full.jpg}

%example table
%\begin{frame}[fragile]
%\frametitle{Basic math operations}
%\begin{center}
%\begin{tabular}{lc} \hline
%\rowcolor{UniBlue!100}col1 & col2 \\ \hline \hline
%\rowcolor{UniBlue!75}3 & 3 \\ \hline
%\rowcolor{UniBlue!90}3 & 3 \\ \hline
%\rowcolor{UniBlue!75}3 & 3 \\ \hline
%\rowcolor{UniBlue!90}3 & 3 \\ \hline
%\end{tabular}
%\end{center}

% items enclosed in square brackets are optional; explanation below
\title[Title1]{File IO}
\subtitle[Title2]{Python for Ecologists}
\author[etal]{Tao Hong, Tom Purucker, Jonathan Flaishans, Marcia Snyder}
\institute[EPA]{
  Ecological Society of America Workshop\\
  Minneapolis, MN\\[1ex]
  \texttt{hongtao510@gmail.com}
}

\begin{document}

%--- the titlepage frame -------------------------%
\begin{frame}[plain]
  \titlepage
\end{frame}

%\begin{frame}
%\frametitle{Frame with reduced font size}
%\Fontvi
%\lipsum[1]
%\end{frame}

%\begin{frame}
%\frametitle{Frame with regular font size}
%\lipsum[1]
%\end{frame}

%\begin{frame}[fragile]
%\frametitle{Generic slide}
%\begin{itemize}
%\item thing 1  
%\item thing 2 
%\item thing 3 
%\item thing 4
%\end{itemize} 
%\end{frame}

\begin{frame}[fragile]
\frametitle{Overview of files}
\begin{itemize}
\item What is a file?
\item Where is a file stored?
\item Reading files
\item Writing files
\item Opening files on the Internet
\item Hairy Issues
\item Problems!
\end{itemize} 
\end{frame}

\begin{frame}[fragile]
\frametitle{What is a file?}
\begin{itemize}
\item Technically - long strings of 0s and 1s
\item Informally - a name for the place your program reads from and writes to
\end{itemize} 
\end{frame}

\begin{frame}[fragile]
\frametitle{Where is a file stored?}
\begin{itemize}
\item Files can be stored on your computer, or on a remote computer.
\begin{itemize}
\item When files are on your computer, life is easy, ex:
\begin{itemize}
\item  'myFileName.txt'
\end{itemize}
\item When files are on a remote computer, you need a specific protocol to retrieve them, ex:
\begin{itemize}
\item 'http://mywebsite.com/myFileName.txt'
\end{itemize}
\end{itemize} 
\end{itemize}
\end{frame}

\begin{frame}[fragile]
\frametitle{How do I get data out of a file?}
\begin{enumerate}
\item Open the file.
  \begin{lstlisting}
  file = open("myFileName.txt")
  \end{lstlisting}
\item Loop through the File (here we loop through)
  \begin{lstlisting}
  for line in file:
        print(line)   # line is a string, lets print it!
  \end{lstlisting}
\item Close the File
  \begin{lstlisting}
  file.close()
  \end{lstlisting}
\end{enumerate}
\end{frame}

\begin{frame}[fragile]
\frametitle{I want to do more than print!}
\begin{itemize}
\item Code:
  \begin{lstlisting}
  file = open('myFileName.txt')
  for line in file:
    print(line)   # line is a string, lets print it!
  file.close()  
  \end{lstlisting}
\item line is a string, so we can do whatever we want to it.  If the line is a sentence, and we wanted to break it up by words, we could do this instead :
  \begin{lstlisting}
file = open('myFileName.txt')
for line in file:
    for word in line.split():
        print(word)   # lets print each word!
file.close()
  \end{lstlisting}
\end{itemize}
\end{frame}

\begin{frame}[fragile]
\frametitle{I want to save a file}
\begin{itemize}
\item We can open a file for writing, as well as reading.
  \begin{lstlisting}
file = open('anotherFile.txt', 'w')
file.write('This is output!\n')
file.close() 
  \end{lstlisting}
\item Be careful not to overwrite an important file!
\end{itemize}
\end{frame}

\begin{frame}[fragile]
\frametitle{How can we open a file from the Internet?}
\begin{itemize}
\item The 'urlopen' function tries to open URLs
\begin{lstlisting}
import urllib2
file = urllib2.urlopen('YourFilePathHere')
\end{lstlisting}
\end{itemize}
\end{frame}

\begin{frame}[fragile]
\frametitle{Why you'll hate files}
\begin{itemize}
\item Files can be open in different modes
\begin{itemize}
\item w - write
\item a - appends
\item r - read (default)
\item and more!
\end{itemize}
\item Files involve string parsing and manipulation
\begin{itemize}
\item This can be a real pain!
\item Corruption prone!
\end{itemize}
\end{itemize} 
\end{frame}

\begin{frame}[fragile]
\frametitle{Exercise}
\begin{enumerate}
\item Read a file
\begin{enumerate}
\item Read StateoftheUnion.txt
\item Output each word on its own line
\end{enumerate}
\item Sum a file
\begin{enumerate}
\item Read IntegerFile.txt, which has many numbers on one line.
\item Calculate the sum, and the mean. Hint, the int() function turns a string into a number.  
\end{enumerate}
\item File copier
\begin{enumerate}
\item Copy http://www.ubertool.org/index.html to a file on your local disk.
\end{enumerate}
\end{enumerate} 
\end{frame}

\end{document}