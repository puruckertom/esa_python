\documentclass{beamer}

\usetheme{default}
\beamertemplatenavigationsymbolsempty

%http://mikedewar.wordpress.com/2009/02/25/latex-beamer-python-beauty/
\definecolor{fore}{RGB}{249,242,215}
%{249,242,215}
\definecolor{back}{RGB}{51,51,51}
%{RGB}{51,51,51}
\definecolor{title}{RGB}{230,96,6}
%{255,0,90}
%http://meyerweb.com/eric/tools/color-blend/
%http://www.census.gov/population/international/data/worldpop/table_population.php
\setbeamercolor{titlelike}{fg=title}
\setbeamercolor{normal text}{fg=fore,bg=back}

\useinnertheme{rectangles}
\definecolor{UniBlue}{RGB}{34,139,34}
\setbeamercolor*{item}{fg=UniBlue}

\usepackage{colortbl}
\usepackage{caption}
\usepackage{listings,bera}
\usepackage{lipsum}
\newcommand\Fontvi{\fontsize{6}{7.2}\selectfont}
\newcommand\Fontvii{\fontsize{7}{7.5}\selectfont}
\newcommand\Fontviii{\fontsize{8}{7.8}\selectfont}
\newcommand\Fontix{\fontsize{9}{8.3}\selectfont}
\definecolor{keywords}{RGB}{255,102,0}
%{255,0,90}
\definecolor{comments}{RGB}{51,153,204}
%\definecolor{comments}{RGB}{83,121,170}
\setbeamertemplate{caption}[numbered]
%\useoutertheme{infolines}
\lstset{language=Python,
keywordstyle=\color{keywords},
commentstyle=\color{comments}\emph}

%\logo{epa-logo-full.jpg}

%example table
%\begin{frame}[fragile]
%\frametitle{Basic math operations}
%\begin{center}
%\begin{tabular}{lc} \hline
%\rowcolor{UniBlue!100}col1 & col2 \\ \hline \hline
%\rowcolor{UniBlue!75}3 & 3 \\ \hline
%\rowcolor{UniBlue!90}3 & 3 \\ \hline
%\rowcolor{UniBlue!75}3 & 3 \\ \hline
%\rowcolor{UniBlue!90}3 & 3 \\ \hline
%\end{tabular}
%\end{center}

% items enclosed in square brackets are optional; explanation below
\title[Title1]{Functions}
\subtitle[Title2]{Python for Ecologists}
\author[etal]{Jon Flaishans, Tom Purucker, Tao Hong, Marcia Snyder}
\institute[EPA]{
  Ecological Society of America Workshop\\
  Minneapolis, MN\\[1ex]
  \texttt{purucker.tom@gmail.com}
}

\begin{document}

%--- the titlepage frame -------------------------%
\begin{frame}[plain]
  \titlepage
\end{frame}

%\begin{frame}
%\frametitle{Frame with reduced font size}
%\Fontvi
%\lipsum[1]
%\end{frame}

%\begin{frame}
%\frametitle{Frame with regular font size}
%\lipsum[1]
%\end{frame}

%\begin{frame}[fragile]
%\frametitle{Generic slide}
%\begin{itemize}
%\item thing 1  
%\item thing 2 
%\item thing 3 
%\item thing 4
%\end{itemize} 
%\end{frame}

\begin{frame}[fragile]
\frametitle{Functions}
\begin{itemize}
  \item A function takes one or more arguments and returns something
  \item Can return any type of data structure, including None
  \item Basic organization of a function:
\item \begin{lstlisting}
def some_function(arg1, arg2):
  #some statements
  return answer
\end{lstlisting}

\end{itemize} 
\end{frame}

\begin{frame}[fragile]
\frametitle{whitespace and naming}
\begin{itemize}
  \item Whitespace usage is important
  \begin{itemize}
  \item Indent consistently with 2 or 4 spaces
  \item Use spaces instead of tabs, -tt
  \end{itemize}
  \item{Naming conventions}
  \begin{itemize}
  \item lower case words, cannot start with a number
  \item use verbs that describe what the function does
  \item underscore\_between\_words
  \end{itemize}
\begin{lstlisting}
def some_function(arg1, arg2):
  #some statements
  return answer
\end{lstlisting}

\end{itemize} 
\end{frame}

\begin{frame}[fragile]
\frametitle{Example Function}
\begin{itemize}
\item Argument names can be informative, helping documentation (e.g., growth\_rate instead of arg1)
\end{itemize}
\begin{lstlisting}
def add_two_numbers(num1, num2):
  return num1 + num2
  
add_two_numbers(6,5)
\end{lstlisting}
\end{frame}

\begin{frame}[fragile]
\frametitle{docstrings}
\begin{itemize}
\item Tripe quoted comment at beginning of functions is accessible as a dunder doc (.\_\_doc\_\_) or help()
\end{itemize}
\begin{lstlisting}
def add_two_numbers(num1, num2):
  """ this function adds two numbers
  together"""
  return num1 + num2
  
help(add_two_numbers)
\end{lstlisting}
\end{frame}

\begin{frame}[fragile]
\frametitle{Types of Arguments}
\begin{itemize}
\item Required arguments do not have a default
\item Keyword arguments can have a default value (and are optional)
\item Keyword arguments are differentiated by setting equal to a value in the function argument list
\end{itemize}
\begin{lstlisting}
def double_it(required1, keyword2=2):
  return required1 * keyword2
  
double_it(6)
double_it(6,3) #actually triples it
\end{lstlisting}
\end{frame}

\end{document}

