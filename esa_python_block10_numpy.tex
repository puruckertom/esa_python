\documentclass{beamer}

\usetheme{default}
\beamertemplatenavigationsymbolsempty

%http://mikedewar.wordpress.com/2009/02/25/latex-beamer-python-beauty/
\definecolor{fore}{RGB}{249,242,215}
%{249,242,215}
\definecolor{back}{RGB}{51,51,51}
%{RGB}{51,51,51}
\definecolor{title}{RGB}{230,96,6}
%{255,0,90}
%http://meyerweb.com/eric/tools/color-blend/
%http://www.census.gov/population/international/data/worldpop/table_population.php
\setbeamercolor{titlelike}{fg=title}
\setbeamercolor{normal text}{fg=fore,bg=back}

\useinnertheme{rectangles}
\definecolor{UniBlue}{RGB}{34,139,34}
\setbeamercolor*{item}{fg=UniBlue}

\usepackage{colortbl}
\usepackage{caption}
\usepackage{listings,bera}
\usepackage{lipsum}
\usepackage{url}
\newcommand\Fontvi{\fontsize{6}{7.2}\selectfont}
\newcommand\Fontvii{\fontsize{7}{7.5}\selectfont}
\newcommand\Fontviii{\fontsize{8}{7.8}\selectfont}
\newcommand\Fontix{\fontsize{9}{8.3}\selectfont}
\definecolor{keywords}{RGB}{255,102,0}
%{255,0,90}
\definecolor{comments}{RGB}{51,153,204}
%\definecolor{comments}{RGB}{83,121,170}
\setbeamertemplate{caption}[numbered]
%\useoutertheme{infolines}
\lstset{language=Python,
keywordstyle=\color{keywords},
commentstyle=\color{comments}\emph}

%\logo{epa-logo-full.jpg}

%example table
%\begin{frame}[fragile]
%\frametitle{Basic math operations}
%\begin{center}
%\begin{tabular}{lc} \hline
%\rowcolor{UniBlue!100}col1 & col2 \\ \hline \hline
%\rowcolor{UniBlue!75}3 & 3 \\ \hline
%\rowcolor{UniBlue!90}3 & 3 \\ \hline
%\rowcolor{UniBlue!75}3 & 3 \\ \hline
%\rowcolor{UniBlue!90}3 & 3 \\ \hline
%\end{tabular}
%\end{center}

% items enclosed in square brackets are optional; explanation below
\title[Title1]{Numpy}
\subtitle[Title2]{Python for Ecologists}
\author[etal]{Tao Hong, Chance Pascale, Tom Purucker}
\institute[EPA]{
  Ecological Society of America Workshop\\
  Minneapolis, MN\\[1ex]
  \texttt{hongtao510@gmail.com}
}

\begin{document}

%--- the titlepage frame -------------------------%
\begin{frame}[plain]
  \titlepage
\end{frame}

%\begin{frame}
%\frametitle{Frame with reduced font size}
%\Fontvi
%\lipsum[1]
%\end{frame}

%\begin{frame}
%\frametitle{Frame with regular font size}
%\lipsum[1]
%\end{frame}

%\begin{frame}[fragile]
%\frametitle{Generic slide}
%\begin{itemize}
%\item thing 1  
%\item thing 2 
%\item thing 3 
%\item thing 4
%\end{itemize} 
%\end{frame}

\begin{frame}[fragile]
\frametitle{Overview}
\begin{itemize}
  \item Install Numpy
  \item Array
  \item Index
  \item Random number
  \item Basic operations
  \item Reference
\end{itemize} 
\end{frame}

\begin{frame}[fragile]
\frametitle{Install Numpy}
\begin{itemize}
  \item Windows
  \begin{itemize}
  \item Official \\
  http://sourceforge.net/projects/numpy/?source=dlp
  \item Unofficial (Windows binaries) \\
  \url{http://www.lfd.uci.edu/~gohlke/pythonlibs/} \\
  \item Portable Python \\
  http://portablepython.com/wiki/PortablePython2.7.5.1/
  \end{itemize} 
\end{itemize} 


\end{frame}

\begin{frame}[fragile]
\frametitle{Arrays}
\begin{itemize}
\item Create an array
\begin{lstlisting}
a = np.array([10, 20, 30, 40], float)
[ 10.  20.  30.  40.]

b = np.array([10, 20, 30, 40])
[10 20 30 40]

c = np. array([[1, 2],[4, 5]])
[[1 2]
 [4 5]]

d = np. array([[1, 2],[4, 5.0]])
[[ 1.  2.]
 [ 4.  5.]]
\end{lstlisting}
\end{itemize}
\end{frame}

\begin{frame}[fragile]
\frametitle{np.arange}
\begin{itemize}
\item np.array is not efficient, try np.arange
\item np.arange(start, stop, step, dtype=None)
\begin{lstlisting}
f = np.arange(0,4,1) #does not include '4'
[0 1 2 3]

g = np.arange(4)
[0 1 2 3]
\end{lstlisting}
\end{itemize}
\end{frame}

\begin{frame}[fragile]
\begin{itemize}
\frametitle{np.linespace}
\item Compare to np.arange
\item np.linspace(start, stop, num, endpoint=True, retstep=False)
\begin{lstlisting}
h = np.linspace(2.0, 3.0, num=5)
[ 2.    2.25  2.5   2.75  3.  ]

i = np.linspace(2.0, 3.0, num=5, endpoint=False)
[ 2.   2.2  2.4  2.6  2.8]

j = np.linspace(2.0, 3.0, num=5, retstep=True)
(array([ 2.  ,  2.25,  2.5 ,  2.75,  3.  ]), 0.25)
\end{lstlisting}
\end{itemize}
\end{frame}

\begin{frame}[fragile]
\frametitle{Other ways to create arrays}
\begin{itemize}
\item np.ones
\begin{lstlisting}
x = np.ones((2,3), float)
[[ 1.  1.  1.]
 [ 1.  1.  1.]]
\end{lstlisting}

\item np.zeros
\begin{lstlisting}
x = np.zeros((2,3), float)
[[ 0.  0.  0.]
 [ 0.  0.  0.]]
\end{lstlisting}

\item np.identity
\begin{lstlisting}
x = np.identity(3, float)
[[ 1.  0.  0.]
 [ 0.  1.  0.]
 [ 0.  0.  1.]]
\end{lstlisting}
\end{itemize}
\end{frame}

\begin{frame}[fragile]
\frametitle{Array properties (1)}
\begin{lstlisting}
a = np.array([[1, 2, 3], [4, 5, 6]], float)
a.shape #shape
(2, 3)

a.reshape(1,6)
[[ 1.  2.  3.  4.  5.  6.]]

a.ndim #number of dimensions
2

a.dtype #data type
float64

a.size #number of elements
6
\end{lstlisting}
\end{frame}

\begin{frame}[fragile]
\frametitle{Array properties (2)}
\begin{lstlisting}
a = np.array([[1, 2, 3], [4, 5, 6]], float)
k = a.flatten()
[ 1.  2.  3.  4.  5.  6.]

l = a.tolist() #array to list
[[1.0, 2.0, 3.0], [4.0, 5.0, 6.0]]
type(l)
<type 'list'>
\end{lstlisting}
\end{frame}

\begin{frame}[fragile]
\frametitle{Array properties (3)}
\begin{lstlisting}
m = a.transpose()
[[ 1.  4.]
 [ 2.  5.]
 [ 3.  6.]]

p = np.array([[1, 2], [3, 4]])
q = np.array([[5, 6], [7,8]])

r = np.concatenate((p,q), axis=0) #Join arrays together
[[1 2]
 [3 4]
 [5 6]
 [7 8]]

s = np.concatenate((p,q), axis=1)
[[1 2 5 6]
 [3 4 7 8]]
\end{lstlisting}
\end{frame}

\begin{frame}[fragile]
\frametitle{Index (1)}
\begin{itemize}
\item Integer index

\begin{lstlisting}
a = np.arange(10)
\end{lstlisting}

\begin{table}[t]
\centering
\begin{tabular}{|l|c|c|c|c|c|c|c|c|c|c|}
\hline
a & 0 & 1 & 2 & 3 & 4 & \cellcolor{red}5 & 6 & 7 & 8 & 9 \\
\hline
\end{tabular}
\end{table}

\begin{lstlisting}
a[5]
5
\end{lstlisting}

\item A range (starts at the 4th and ends before 6th)
\begin{table}[t]
\centering
\begin{tabular}{|l|c|c|c|c|c|c|c|c|c|c|}
\hline
a & 0 & 1 & 2 &\cellcolor{red} 3 &\cellcolor{red} 4 & 5 & 6 & 7 & 8 & 9 \\
\hline
\end{tabular}
\end{table}

\begin{lstlisting}
a[3:5]
[3, 4]
\end{lstlisting}
\end{itemize}
\end{frame}

\begin{frame}[fragile]
\frametitle{Index (2)}
\begin{itemize}
\item The first three elements
\begin{table}[t]
\centering
\begin{tabular}{|l|c|c|c|c|c|c|c|c|c|c|}
\hline
a &\cellcolor{red} 0 &\cellcolor{red} 1 &\cellcolor{red} 2 & 3 & 4 & 5 & 6 & 7 & 8 & 9 \\
\hline
\end{tabular}
\end{table}
\begin{lstlisting}
a[:3]
[0,1,2]
\end{lstlisting}
\item Counting backwards
\begin{table}[t]
\centering
\begin{tabular}{|l|c|c|c|c|c|c|c|c|c|c|}
\hline
a &\cellcolor{red} 0 &\cellcolor{red} 1 &\cellcolor{red} 2 & 3 & 4 & 5 & 6 & 7 & 8 & 9 \\
\hline
\end{tabular}
\end{table}
\begin{lstlisting}
a[:-7]
[0,1,2]
\end{lstlisting}
\end{itemize}
\end{frame}

\begin{frame}[fragile]
\frametitle{Index (3)}
\begin{itemize}
\item Reverse the array
\begin{table}[t]
\centering
\begin{tabular}{|l|c|c|c|c|c|c|c|c|c|c|}
\hline
a & 9 & 8 & 7 & 6 & 5 & 4 & 3 & 2 & 1 & 0 \\
\hline
\end{tabular}
\end{table}
\begin{lstlisting}
a[::-1]
[9, 8, 7, 6, 5, 4, 3, 2, 1, 0]
\end{lstlisting}

\item How to get?
\begin{table}[t]
\centering
\begin{tabular}{|l|c|c|c|c|c|c|c|c|c|c|}
\hline
a &\cellcolor{red} 9 & 8 & 7 &\cellcolor{red} 6 & 5 & 4 &\cellcolor{red} 3 & 2 & 1 &\cellcolor{red} 0 \\
\hline
\end{tabular}
\end{table}

\begin{lstlisting}
a[::-3]
[9, 6, 3, 0]
\end{lstlisting}

\end{itemize}
\end{frame}

\begin{frame}[fragile]
\frametitle{Index (4)}
\begin{itemize}
\item Modify content
\begin{table}[t]
\centering
\begin{tabular}{|l|c|c|c|c|c|c|c|c|c|c|}
\hline
a & 0 & 1 &\cellcolor{red} 100 &\cellcolor{red} 101 & 4 & 5 & 6 & 7 & 8 & 9 \\
\hline
\end{tabular}
\end{table}
\begin{lstlisting}
a[2:4]=[100,101]
[  0   1 100 101   4   5   6   7   8   9]
\end{lstlisting}

\item a[i:j:k] \\ 
i-first, j-last (not included), k-step
\begin{table}[t]
\centering
\begin{tabular}{|l|c|c|c|c|c|c|c|c|c|c|}
\hline
a & 0 &\cellcolor{red} 1 & 100 &\cellcolor{red} 101 & 4 &\cellcolor{red} 5 & 6 &\cellcolor{red} 7 & 8 & 9 \\
\hline
\end{tabular}
\end{table}

\begin{lstlisting}
a[1:-1:2]
[  1 101   5   7]
\end{lstlisting}
\item Try 
\begin{lstlisting}
a[1::2]
[  1 101   5   7   9]
\end{lstlisting}
\end{itemize}
\end{frame}

\begin{frame}[fragile]
\frametitle{Index (5)}
\begin{itemize}
\item Find index of an array
\begin{lstlisting}
x = np.arange(9.).reshape(3, 3)
x

[[ 0.  1.  2.]
 [ 3.  4.  5.]
 [ 6.  7.  8.]]

np.where( x > 4.5 )
(array([1, 2, 2, 2]), array([2, 0, 1, 2]))
\end{lstlisting}
\end{itemize}
\end{frame}

\begin{frame}[fragile]
\frametitle{Random number (1)}
\begin{itemize}
\item rand(dim) Uniform distribution over [0, 1)
\begin{lstlisting}
a = np.random.rand(2, 3)
[[ 0.42811767  0.43032497  0.19511638]
 [ 0.19985235  0.09149539  0.42384995]]
\end{lstlisting}
\item randn(dim) standard normal
\begin{lstlisting}
a = np.random.randn(2, 3)
[[ 0.41391683  1.31774009 -1.10235464]
 [ 0.32073693  0.32847825 -0.49657114]]
\end{lstlisting}
\end{itemize}
\end{frame}

\begin{frame}[fragile]
\frametitle{Random number (2)}
\begin{itemize}
\item log-normal \\
lognormal(mean, sigma, dim)
\item Possion \\ 
poisson(mean, dim)
\item Beta \\ 
beta(a, b, dim)
\item Fix a seed \\ 
seed(number)
\item more distritions are available \\
\url{http://docs.scipy.org/doc/numpy/reference/routines.random.html}
\end{itemize}
\end{frame}

\begin{frame}[fragile]
\frametitle{Basic operations (1)}
\begin{itemize}
\item sum
\begin{lstlisting}
a = np.array([[1, 2, 3], [4, 5, 6]], float)
[[ 1.  2.  3.]
 [ 4.  5.  6.]]
a.sum()
21.0
a.sum(axis=0) #col sum
[ 5.  7.  9.]
\end{lstlisting}
\item mean
\begin{lstlisting}
a.mean()
3.5
\end{lstlisting}
\item variance
\begin{lstlisting}
a.var()
2.91666666667
\end{lstlisting}
\end{itemize}
\end{frame}

\begin{frame}[fragile]
\frametitle{Basic operations (2)}
\begin{itemize}
\item min
\begin{lstlisting}
a = np.array([[1, 2, 2], [4, 5, 4]], float)
[[ 1.  2.  2.]
 [ 4.  5.  4.]]
a.min()
1.0
\end{lstlisting}
\item index lookup
\begin{lstlisting}
a.argmin()
0
\end{lstlisting}
\item find unique elements
\begin{lstlisting}
np.unique(a)
[ 1.  2.  4.  5.]
\end{lstlisting}
\item diagnoal
\begin{lstlisting}
a.diagonal()
[ 1.  5.]
\end{lstlisting}
\end{itemize}
\end{frame}

\begin{frame}[fragile]
\frametitle{Basic operations (3)}
\begin{itemize}
\item inverse
\begin{lstlisting}
a = np.array([[1, 2], [4, 5]], float)
[[ 1.  2.]
 [ 4.  5.]]

b=np.linalg.inv(a)
[[-1.66666667  0.66666667]
 [ 1.33333333 -0.33333333]]
\end{lstlisting}
\item determinant
\begin{lstlisting}
np.linalg.det(a)
-3
\end{lstlisting}
\end{itemize}
\end{frame}

\begin{frame}[fragile]
\frametitle{Basic operations (4)}
\begin{itemize}
\item matrix multiply
\begin{lstlisting}
np.dot(a,b)
[[ 1.  0.]
 [ 0.  1.]]
\end{lstlisting}
\item element-wise multiply
\begin{lstlisting}
a*b
[[-1.66666667  1.33333333]
 [ 5.33333333 -1.66666667]]
\end{lstlisting}
\item solve a linear system \\
a x c=b
\begin{lstlisting}
c=np.linalg.solve(a,b)
[[ 3.66666667 -1.33333333]
 [-2.66666667  1.        ]]
\end{lstlisting}
\end{itemize}
\end{frame}

\begin{frame}[fragile]
\frametitle{Shallow copy}
\begin{itemize}
\item arrays share the same elements
\begin{lstlisting}
a = np.arange(0, 60, 10)
b = a
a
[ 0 10 20 30 40 50]
b
[ 0 10 20 30 40 50]

a[0]=100
a
[100  10  20  30  40  50]
b
[100  10  20  30  40  50]
\end{lstlisting}
\end{itemize}
\end{frame}

\begin{frame}[fragile]
\frametitle{Deep copy}
\begin{itemize}
\item each array has its own elements
\begin{lstlisting}
a = np.arange(0, 60, 10)
import copy
b= copy.deepcopy(a)
a
[ 0 10 20 30 40 50]
b
[ 0 10 20 30 40 50]

a[0]=100
a
[100  10  20  30  40  50]
b
[0  10  20  30  40  50]
\end{lstlisting}
\end{itemize}
\end{frame}

\begin{frame}[fragile]
\frametitle{Reference}
\begin{itemize}
\item Official document \\
\url{http://docs.scipy.org/doc/}
\item NumPy for MATLAB users \\
\url{http://www.scipy.org/NumPy_for_Matlab_Users}
\item NumPy for R (and S-Plus) users \\
\url{http://mathesaurus.sourceforge.net/r-numpy.html}
\item Stackoverflow \\
\url{http://stackoverflow.com/}
\end{itemize}
\end{frame}

\end{document}