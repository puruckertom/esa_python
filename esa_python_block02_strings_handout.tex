\documentclass{beamer}
\usepackage{handoutWithNotes}
\pgfpagesuselayout{4 on 1 with notes}[a4paper,border shrink=5mm]

\usetheme{default}
\beamertemplatenavigationsymbolsempty

%http://mikedewar.wordpress.com/2009/02/25/latex-beamer-python-beauty/
%\definecolor{fore}{RGB}{249,242,215}
\definecolor{fore}{RGB}{51,51,51}
%{249,242,215}
%\definecolor{back}{RGB}{51,51,51}
\definecolor{back}{RGB}{249,242,215}
%{RGB}{51,51,51}
\definecolor{title}{RGB}{230,96,6}
%{255,0,90}
%http://meyerweb.com/eric/tools/color-blend/
%http://www.census.gov/population/international/data/worldpop/table_population.php
\setbeamercolor{titlelike}{fg=title}
\setbeamercolor{normal text}{fg=fore,bg=back}

\useinnertheme{rectangles}
\definecolor{UniBlue}{RGB}{34,139,34}
\setbeamercolor*{item}{fg=UniBlue}

\usepackage{colortbl}
\usepackage{caption}
\usepackage{listings,bera}
\usepackage{lipsum}
\newcommand\Fontv{\fontsize{5}{6.2}\selectfont}
\newcommand\Fontvi{\fontsize{6}{7.2}\selectfont}
\newcommand\Fontvii{\fontsize{7}{7.5}\selectfont}
\newcommand\Fontviii{\fontsize{8}{8.1}\selectfont}
\newcommand\Fontix{\fontsize{9}{8.3}\selectfont}
\definecolor{keywords}{RGB}{255,102,0}
%{255,0,90}
\definecolor{comments}{RGB}{51,153,204}
%\definecolor{comments}{RGB}{83,121,170}
\setbeamertemplate{caption}[numbered]
%\useoutertheme{infolines}
\lstset{language=Python,
keywordstyle=\color{keywords},
commentstyle=\color{comments}\emph}

%\logo{epa-logo-full.jpg}

%example table
%\begin{frame}[fragile]
%\frametitle{Basic math operations}
%\begin{center}
%\begin{tabular}{lc} \hline
%\rowcolor{UniBlue!100}col1 & col2 \\ \hline \hline
%\rowcolor{UniBlue!75}3 & 3 \\ \hline
%\rowcolor{UniBlue!90}3 & 3 \\ \hline
%\rowcolor{UniBlue!75}3 & 3 \\ \hline
%\rowcolor{UniBlue!90}3 & 3 \\ \hline
%\end{tabular}
%\end{center}

% items enclosed in square brackets are optional; explanation below
\title[Title1]{Strings}
\subtitle[Title2]{Python for Ecologists}
\author[etal]{Tom Purucker, Tao Hong, Chance Pascale}
\institute[EPA]{
  Ecological Society of America Workshop\\
  Portland, OR\\[1ex]
  \texttt{purucker.tom@gmail.com}
}

\begin{document}

%--- the titlepage frame -------------------------%
\begin{frame}[plain]
  \titlepage
\end{frame}

%\begin{frame}
%\frametitle{Frame with reduced font size}
%\Fontvi
%\lipsum[1]
%\end{frame}

%\begin{frame}
%\frametitle{Frame with regular font size}
%\lipsum[1]
%\end{frame}

%\begin{frame}[fragile]
%\frametitle{Generic slide}
%\begin{itemize}
%\item thing 1  
%\item thing 2 
%\item thing 3 
%\item thing 4
%\end{itemize} 
%\end{frame}

\begin{frame}[fragile]
\frametitle{Accessing documentation in Python}
\begin{itemize}
  \item Use dir() and help() commands to access documentation
\begin{lstlisting}
import sys
dir(sys)
help(sys)
help(sys.argv)
help(str)
dir(str)
help(len)
\end{lstlisting} 
\item or just Google it -- \href{http://www.google.com/webhp?sourceid=chrome-instant&ie=UTF-8#hl=en&output=search&sclient=psy-ab&q=python\%20len&oq=&gs_l=&pbx=1&fp=175fca51e4ec0d4b&bav=on.2,or.r_gc.r_pw.r_qf.&biw=1292&bih=806}{'python len'}
\end{itemize} 
\end{frame}

\begin{frame}[fragile]
\frametitle{dunder methods}
\begin{itemize}
\item "dunder" is short for double underline, e.g., \_\_init\_\_ 
\item also known as special or magic methods 
\item + means different things for numbers and for strings
\begin{lstlisting}
3 + 5
"Hello " + "world"
\end{lstlisting}
\item \_\_add\_\_ for each type handles the different cases
\end{itemize} 
\end{frame}

\begin{frame}[fragile]
\frametitle{len v len()}
\begin{itemize}
\item len - refers to the object 
\begin{lstlisting}
help(len)
\end{lstlisting}
\item len() calls the function with a supplied argument
\begin{lstlisting}
len('Geospiza magnirostris')
\end{lstlisting}
\end{itemize} 
\end{frame}

\begin{frame}[fragile]
\frametitle{Single and Double quotes}
\begin{itemize}
\item No preference between single and double quotes
\begin{lstlisting}
aint = "ain't"
aint = 'ain"t'
\end{lstlisting}
\end{itemize} 
\end{frame}

\begin{frame}[fragile]
\frametitle{Strings}
\Fontix
\begin{itemize}
\item Defining a string - single quote
\begin{lstlisting}
darwins_finch = 'Geospiza magnirostris'
darwins_finch
\end{lstlisting}
\item Quoting within the string - double quote
\begin{lstlisting}
darwin_quote ="'Great is the power of steady misrepresentation'"
darwin_quote
darwin_quote.lower()
darwin_quote[13:18]
darwin_quote[0] #Python is zero-based
\end{lstlisting}
\item #Slicing strings- extracts up to, but not including the second index
\begin{lstlisting}
darwin_quote[0:1] 
darwin_quote[0:6]
darwin_quote[:12]
darwin_quote[12:]
darwin_quote[:12] + darwin_quote[12:]
\end{lstlisting}
\end{itemize}
\end{frame}

\begin{frame}[fragile]
\frametitle{Multi-line strings}
\Fontix
\begin{itemize}
\item Multiline strings - triple quote
\begin{lstlisting}
long_darwin_quote = '''There is grandeur in this view
of life, with its several powers, having been originally 
breathed into a few forms or into one; and that, whilst 
this planet has gone cycling on according to the fixed 
law of gravity, from so simple a beginning endless 
forms most beautiful and most wonderful have been, and 
are being, evolved.'''
long_darwin_quote
print(long_darwin_quote)
len(long_darwin_quote)
\end{lstlisting}
\end{itemize}
\end{frame}

\begin{frame}[fragile]
\frametitle{C-like string formatting}
\begin{lstlisting}
>>> "%s %s" % ('Hello','Portland')
'Hello Portland'
\end{lstlisting}
\end{frame}

\begin{frame}[fragile]
\frametitle{Exercise 2- Run the script exer02\_strings.py}
\Fontvi
\begin{lstlisting}
class TestStrings(unittest.TestCase):
    def test_strings(self):
        # Create the variable ``hola`` and assign 'hello world'
        #**************************************************

        self.assertEqual(hola, """hello world""")
        self.assert_(isinstance(hola, str))

        # Create a string, ``hola2`` that equals ``hola`` multiplied by 2
        #**************************************************

        self.assertEqual(hola2, 'hello worldhello world')
.
        # Create a triple quoted string
        # ``darwin_quote2`` that has the following content:
        # Darwin said, "There is grandeur in this view of things."
        #**************************************************

        self.assertEqual(darwin_quote2, 'Darwin said, "There is grandeur in this view of things."')

        # Assign the method names of a string to a variable ``string_methods``
        # use ``dir()`` to list them
        #**************************************************

        self.assertEqual(string_methods, ['__add__', ...])

        # Create a variable where_is_gra that has holds the index of the
        # substring "gra" in the string darwin_quote2.  (Find a string method to
        # figure it out)
        #**************************************************

        self.assertEqual(where_is_gra, 23)

\end{lstlisting}
\end{frame}

\end{document}