\documentclass{beamer}

\usetheme{default}
\beamertemplatenavigationsymbolsempty

%http://mikedewar.wordpress.com/2009/02/25/latex-beamer-python-beauty/
\definecolor{fore}{RGB}{249,242,215}
%{249,242,215}
\definecolor{back}{RGB}{51,51,51}
%{RGB}{51,51,51}
\definecolor{title}{RGB}{230,96,6}
%{255,0,90}
%http://meyerweb.com/eric/tools/color-blend/
%http://www.census.gov/population/international/data/worldpop/table_population.php
\setbeamercolor{titlelike}{fg=title}
\setbeamercolor{normal text}{fg=fore,bg=back}

\useinnertheme{rectangles}
\definecolor{UniBlue}{RGB}{34,139,34}
\setbeamercolor*{item}{fg=UniBlue}

\usepackage{colortbl}
\usepackage{caption}
\usepackage{listings,bera}
\usepackage{lipsum}
\newcommand\Fontvi{\fontsize{6}{7.2}\selectfont}
\newcommand\Fontvii{\fontsize{7}{7.5}\selectfont}
\newcommand\Fontviii{\fontsize{8}{7.8}\selectfont}
\newcommand\Fontix{\fontsize{9}{8.3}\selectfont}
\definecolor{keywords}{RGB}{255,102,0}
%{255,0,90}
\definecolor{comments}{RGB}{51,153,204}
%\definecolor{comments}{RGB}{83,121,170}
\setbeamertemplate{caption}[numbered]
%\useoutertheme{infolines}
\lstset{language=Python,
keywordstyle=\color{keywords},
commentstyle=\color{comments}\emph}

%\logo{epa-logo-full.jpg}

%example table
%\begin{frame}[fragile]
%\frametitle{Basic math operations}
%\begin{center}
%\begin{tabular}{lc} \hline
%\rowcolor{UniBlue!100}col1 & col2 \\ \hline \hline
%\rowcolor{UniBlue!75}3 & 3 \\ \hline
%\rowcolor{UniBlue!90}3 & 3 \\ \hline
%\rowcolor{UniBlue!75}3 & 3 \\ \hline
%\rowcolor{UniBlue!90}3 & 3 \\ \hline
%\end{tabular}
%\end{center}

% items enclosed in square brackets are optional; explanation below
\title[Title1]{Loops}
\subtitle[Title2]{Python for Ecologists}
\author[etal]{Chance Pascale, Tom Purucker, Tao Hong}
\institute[EPA]{
  Ecological Society of America Workshop\\
  Portland, OR\\[1ex]
  \texttt{purucker.tom@gmail.com}
}

\begin{document}

%--- the titlepage frame -------------------------%
\begin{frame}[plain]
  \titlepage
\end{frame}

%\begin{frame}
%\frametitle{Frame with reduced font size}
%\Fontvi
%\lipsum[1]
%\end{frame}

%\begin{frame}
%\frametitle{Frame with regular font size}
%\lipsum[1]
%\end{frame}

%\begin{frame}[fragile]
%\frametitle{Generic slide}
%\begin{itemize}
%\item thing 1  
%\item thing 2 
%\item thing 3 
%\item thing 4
%\end{itemize} 
%\end{frame}

\begin{frame}[fragile]
\frametitle{Conditionals}
\begin{itemize}
\item whitespace is very important, determines the close
\end{itemize}
\begin{lstlisting}
if grade >= 90:
  print "A"
elif grade >= 80:
  print "B"
elif grade >= 70:
  print "C"
elif grade >= 60:
  print "D"
else:
  print "F"  
\end{lstlisting} 
\end{frame}

\begin{frame}[fragile]
\frametitle{Booleans}
\begin{lstlisting}
cloudy = True
rainy = False

# Testing booleans
print(bool("alpha"<"beta"))
True
\end{lstlisting}
\begin{itemize}
\item Strings are compared alphabetically when sorted
\end{itemize}
\end{frame}

\begin{frame}[fragile]
\frametitle{Comparison operators}
\begin{itemize}
\item Python comparison operators
\end{itemize}
\begin{lstlisting}
# x == y      x equals y
# x < y       x is less than y
# x > y       x is greater than y
# x >= y      x is greater than or equal to y
# x <= y      x is less than or equal to y
# x != y      x is not equal to y
# x is y      x and y are the same object
# x is not y  x and y are different objects
# x in y      x is a member of the container y
# x not in y  x is not a member of the container
\end{lstlisting}
\end{frame}

\begin{frame}[fragile]
\frametitle{Loop iteration}
\begin{lstlisting}
# through a list of numbers
for value in [1,2,3,4,5,6,7]:
  print value
# using range
for value in range(1,8):
 print value
\end{lstlisting}
\end{frame}

\begin{frame}[fragile]
\frametitle{enumeration}
\begin{lstlisting}
# through a list of strings with indices
watersheds = ["Suwanne", "Oconee", "Tennessee", "Flint"]
for index, value in enumerate(watersheds):
  print index, value
# can also loop through dictionaries
\end{lstlisting}
\end{frame}

\begin{frame}[fragile]
\frametitle{break, continue, and pass}
\begin{lstlisting}
# using break
for number in range(1,8):
 if number < 5:
   print number
 else:
     break
# using continue
for number in range(1,8):
 if number < 5:
   print number
   continue
# pass
for number in range(1,8):
  if number < 5:
    print number
  else:
    pass
\end{lstlisting}
\end{frame}


\end{document}