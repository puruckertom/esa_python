\documentclass{beamer}
\usepackage{handoutWithNotes}
\pgfpagesuselayout{4 on 1 with notes}[a4paper,border shrink=5mm]

\usetheme{default}
\beamertemplatenavigationsymbolsempty

%http://mikedewar.wordpress.com/2009/02/25/latex-beamer-python-beauty/
%\definecolor{fore}{RGB}{249,242,215}
\definecolor{fore}{RGB}{51,51,51}
%{249,242,215}
%\definecolor{back}{RGB}{51,51,51}
\definecolor{back}{RGB}{249,242,215}
%{RGB}{51,51,51}
\definecolor{title}{RGB}{230,96,6}
%{255,0,90}
%http://meyerweb.com/eric/tools/color-blend/
%http://www.census.gov/population/international/data/worldpop/table_population.php
\setbeamercolor{titlelike}{fg=title}
\setbeamercolor{normal text}{fg=fore,bg=back}

\useinnertheme{rectangles}
\definecolor{UniBlue}{RGB}{34,139,34}
\setbeamercolor*{item}{fg=UniBlue}

\usepackage{colortbl}
\usepackage{caption}
\usepackage{listings,bera}
\usepackage{lipsum}
\newcommand\Fontvi{\fontsize{6}{7.2}\selectfont}
\newcommand\Fontvii{\fontsize{7}{7.5}\selectfont}
\newcommand\Fontviii{\fontsize{8}{7.8}\selectfont}
\newcommand\Fontix{\fontsize{9}{8.3}\selectfont}
\definecolor{keywords}{RGB}{255,102,0}
%{255,0,90}
\definecolor{comments}{RGB}{51,153,204}
%\definecolor{comments}{RGB}{83,121,170}
\setbeamertemplate{caption}[numbered]
%\useoutertheme{infolines}
\lstset{language=Python,
keywordstyle=\color{keywords},
commentstyle=\color{comments}\emph}

%\logo{epa-logo-full.jpg}

%example table
%\begin{frame}[fragile]
%\frametitle{Basic math operations}
%\begin{center}
%\begin{tabular}{lc} \hline
%\rowcolor{UniBlue!100}col1 & col2 \\ \hline \hline
%\rowcolor{UniBlue!75}3 & 3 \\ \hline
%\rowcolor{UniBlue!90}3 & 3 \\ \hline
%\rowcolor{UniBlue!75}3 & 3 \\ \hline
%\rowcolor{UniBlue!90}3 & 3 \\ \hline
%\end{tabular}
%\end{center}

% items enclosed in square brackets are optional; explanation below
\title[Title1]{Classes}
\subtitle[Title2]{Python for Ecologists}
\author[etal]{Tom Purucker, Tao Hong, Chance Pascale}
\institute[EPA]{
  Ecological Society of America Workshop\\
  Portland, OR\\[1ex]
  \texttt{purucker.tom@gmail.com}
}

\begin{document}

%--- the titlepage frame -------------------------%
\begin{frame}[plain]
  \titlepage
\end{frame}

%\begin{frame}
%\frametitle{Frame with reduced font size}
%\Fontvi
%\lipsum[1]
%\end{frame}

%\begin{frame}
%\frametitle{Frame with regular font size}
%\lipsum[1]
%\end{frame}

%\begin{frame}[fragile]
%\frametitle{Generic slide}
%\begin{itemize}
%\item thing 1  
%\item thing 2 
%\item thing 3 
%\item thing 4
%\end{itemize} 
%\end{frame}
\begin{frame}[fragile]
\frametitle{Python objects}
\begin{itemize}
\item Everything in Python is an object with these properties
\begin{enumerate}  
  \item an identity (id) 
  \item a type (type)
  \item a value (mutable or immutable)
\end{enumerate}
\end{itemize} 
\end{frame}

\begin{frame}[fragile]
\frametitle{Each Python object has an id}
\begin{lstlisting}
>>> n_predators = 12
>>> id(n_predators)
4298191056
\end{lstlisting} 
\end{frame}

\begin{frame}[fragile]
\frametitle{Each Python object has a type}
\begin{lstlisting}
>>> n_predators = 12
>>> type(n_predators)
<type 'int'>
\end{lstlisting}
\end{frame}

\begin{frame}[fragile]
\frametitle{Each Python object has a value}
\begin{itemize}
\item String, integer, and tuple object values are \emph{immutable}
\begin{lstlisting}
>>> n_prey = 88
>>> id(n_prey)
4298193184
>>> n_prey = 96
>>> id(n_prey)
4298192992 # id for n_prey has changed
\end{lstlisting}
\item Dictionary and list items are \emph{mutable}
\begin{lstlisting}
>>> birds = ["cardinal", "oriole"]
>>> id(birds)
4332756000
>>> birds.append("gnatcatcher")
>>> id(birds)
4332756000 # id is still the same
\end{lstlisting}
\end{itemize}
\end{frame}

\begin{frame}[fragile]
\frametitle{Classes}
\begin{itemize}
\item Classes consist of
\begin{itemize}
\item collections of data structures
\item collections of methods (functions)
\end{itemize}
\item Class methods typically operate on the data structures of the class
\item Class users then call methods and do not have to manipulate the data
\end{itemize}
\end{frame}

\begin{frame}[fragile]
\frametitle{self variable}
\begin{itemize}
\item A class instance refers to itself as 'self'
\item All methods require self as the first argument/parameter inside the class
\item But users of the class do not include it in calls to the methods 
\item All data and methods calls are preceded by self within the class (e.g., self.age() or self.find\_integral(some arguments...)
\end{itemize}
\end{frame}

\begin{frame}[fragile]
\frametitle{Creating a class}
\begin{itemize}
\item object is the base class
\item dunder init is a constructor
\item all methods take self as the first argument/parameter
\end{itemize}
\end{frame}

\begin{frame}[fragile]
\frametitle{Code for creating a class}
\begin{lstlisting}
#create the Rabbit class, starts with 10 hit points
class Rabbit(object):
  def __init__(self, name):
    self.name = name
    self.hit_points = 10
        
  def hop(self):
    self.hit_points = self.hit_points - 1
    print "%s hops one node, now has %i hit points." 
    % (self.name, self.hit_points)
    
  def eat_carrot(self):
    self.hit_points = self.hit_points + 3
    print "%s munches a carrot, now has %i hit points." 
    % (self.name, self.hit_points)
\end{lstlisting} 
\end{frame}

\begin{frame}[fragile]
\frametitle{Code to create some rabbits}
\begin{itemize}
\item We can now create objects of Rabbit class and give them names
\end{itemize}
\begin{lstlisting}
#create some Rabbits
were = Rabbit("Were-Rabbit")
harvey = Rabbit("Harvey Rabbit")
jessica = Rabbit("Jessica Rabbit")
dir(jessica)
\end{lstlisting}
\end{frame}

\begin{frame}[fragile]
\frametitle{Code to create some rabbits}
\begin{itemize}
\item We can now create objects of Rabbit class and give them names
\end{itemize}
\begin{lstlisting}
#create some Rabbits
were = Rabbit("Were-Rabbit")
harvey = Rabbit("Harvey Rabbit")
jessica = Rabbit("Jessica Rabbit")
dir(jessica)
\end{lstlisting}
\end{frame}

\begin{frame}[fragile]
\frametitle{Call the methods of the created rabbits}
\begin{itemize}
\item We can now create objects of Rabbit class and give them names
\end{itemize}
\begin{lstlisting}
#Rabbits hop around and eat carrots
were.hop()
jessica.eat_carrot()
harvey.hop()
jessica.hop()
were.eat_carrot()
\end{lstlisting}
\end{frame}

\begin{frame}[fragile]
\frametitle{Create a frog subclass}
\Fontix
\begin{itemize}
\item Subclasses can inherit the data and methods of the original class and extend them
\end{itemize}
\begin{lstlisting}
#Create a Frog class that extends the rabbit class
class Frog(Rabbit):
    # create a new croak method
    def croak(self):
        self.hit_points = self.hit_points - 1
        print "%s croaks, now has %i hit points." 
        % (self.name, self.hit_points)
   # override the eat_carrot method
    def eat_carrot(self):
        print "%s cannot eat a carrot, it is too big!." 
        % (self.name)
   # create an eat_fly method
    def eat_fly(self):
        self.hit_points = self.hit_points + 2
        print "%s eats a fly, now has %i hit points." 
        % (self.name, self.hit_points)
\end{lstlisting}
\end{frame}

\begin{frame}[fragile]
\frametitle{Create Frog objects and call its methods}
\begin{lstlisting}
## Create a frog
frogger = Frog("Frogger")
# Do frog stuff
frogger.croak()
frogger.eat_carrot()
frogger.eat_fly()
frogger.hop()
\end{lstlisting}
\end{frame}

\end{document}