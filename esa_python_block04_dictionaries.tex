\documentclass{beamer}

\usetheme{default}
\beamertemplatenavigationsymbolsempty

%http://mikedewar.wordpress.com/2009/02/25/latex-beamer-python-beauty/
\definecolor{fore}{RGB}{249,242,215}
%{249,242,215}
\definecolor{back}{RGB}{51,51,51}
%{RGB}{51,51,51}
\definecolor{title}{RGB}{230,96,6}
%{255,0,90}
%http://meyerweb.com/eric/tools/color-blend/
%http://www.census.gov/population/international/data/worldpop/table_population.php
\setbeamercolor{titlelike}{fg=title}
\setbeamercolor{normal text}{fg=fore,bg=back}

\useinnertheme{rectangles}
\definecolor{UniBlue}{RGB}{34,139,34}
\setbeamercolor*{item}{fg=UniBlue}

\usepackage{colortbl}
\usepackage{caption}
\usepackage{listings,bera}
\usepackage{lipsum}
\newcommand\Fontvi{\fontsize{6}{7.2}\selectfont}
\newcommand\Fontvii{\fontsize{7}{7.5}\selectfont}
\newcommand\Fontviii{\fontsize{8}{7.8}\selectfont}
\newcommand\Fontix{\fontsize{9}{8.3}\selectfont}
\definecolor{keywords}{RGB}{255,102,0}
%{255,0,90}
\definecolor{comments}{RGB}{51,153,204}
%\definecolor{comments}{RGB}{83,121,170}
\setbeamertemplate{caption}[numbered]
%\useoutertheme{infolines}
\lstset{language=Python,
keywordstyle=\color{keywords},
commentstyle=\color{comments}\emph}

%\logo{epa-logo-full.jpg}

%example table
%\begin{frame}[fragile]
%\frametitle{Basic math operations}
%\begin{center}
%\begin{tabular}{lc} \hline
%\rowcolor{UniBlue!100}col1 & col2 \\ \hline \hline
%\rowcolor{UniBlue!75}3 & 3 \\ \hline
%\rowcolor{UniBlue!90}3 & 3 \\ \hline
%\rowcolor{UniBlue!75}3 & 3 \\ \hline
%\rowcolor{UniBlue!90}3 & 3 \\ \hline
%\end{tabular}
%\end{center}

% items enclosed in square brackets are optional; explanation below
\title[Title1]{Dictionaries}
\subtitle[Title2]{Python for Ecologists}
\author[etal]{Tao Hong, Chance Pascale, Tom Purucker}
\institute[EPA]{
  Ecological Society of America Workshop\\
  Minneapolis, MN\\[1ex]
  \texttt{hongtao510@gmail.com}
}

\begin{document}

%--- the titlepage frame -------------------------%
\begin{frame}[plain]
  \titlepage
\end{frame}

%\begin{frame}
%\frametitle{Frame with reduced font size}
%\Fontvi
%\lipsum[1]
%\end{frame}

%\begin{frame}
%\frametitle{Frame with regular font size}
%\lipsum[1]
%\end{frame}

%\begin{frame}[fragile]
%\frametitle{Generic slide}
%\begin{itemize}
%\item thing 1  
%\item thing 2 
%\item thing 3 
%\item thing 4
%\end{itemize} 
%\end{frame}

\begin{frame}[fragile]
\frametitle{Dictionaries}
\begin{itemize}
  \item Also known as associative arrays or hashmaps
  \item {key:value}
  \item Are mutable, like lists
  \item Unlike lists, index can be something other than an integer 
  \item Lists keep order, dictionaries don't
  \item Can be very efficient for searching and for table lookups
  \begin{lstlisting}
    bw_grams = {}
    bw_grams['Spring peeper'] = 4
    bw_grams['Bullfrog'] = 500
    bw_grams['Cane toad'] = 1800
    print bw_grams['Bullfrog']
  \end{lstlisting} 
\end{itemize} 
\end{frame}

\begin{frame}[fragile]
\frametitle{Setting keys and values}
\begin{lstlisting}
print bw_grams['Barking treefrog']
'Barking treefrog' in bw_grams
bw_grams.has_key('Barking treefrog')
print bw_grams.get('Barking treefrog', 'Not found')
# Compare
bw_grams.setdefault('Barking treefrog', 80) 
# another way to set a default value for a key
if 'Barking treefrog' not in bw_grams:
  bw_grams['Barking treefrog'] = 80

\end{lstlisting}
\end{frame}

\begin{frame}[fragile]
\frametitle{Mixing types}
\begin{itemize}
\item Can be used to track properties of individuals in an individual-based model
\begin{lstlisting}
male43 = {"sp":"Orca", "bw":10., "status":"suscept"}
male43["status"] = "infected"
male43
\end{lstlisting}
\item Dictionary name (e.g., male43) can be nested and itself be a key
\begin{lstlisting}
male4 = {"male43" : male43}
male4
male4['male43']["status"]
\end{lstlisting}
\end{itemize}
\end{frame}

\begin{frame}[fragile]
\frametitle{Using variables to map dictionaries}
\begin{lstlisting}
bw_grams = {}
frog = 'Cricket frog'
weight = '10'
bw_grams['frog'] = frog
bw_grams['weight'] = weight
\end{lstlisting}
\end{frame}

\begin{frame}[fragile]
\begin{itemize}
\frametitle{Deleting a key}
\item Compare
\begin{lstlisting}
del bw_grams['infected']
bw_grams.pop('infected')
\end{lstlisting}
\end{itemize}
\end{frame}

\begin{frame}[fragile]
\begin{itemize}
\frametitle{Other methods}
\item List all contents
\begin{lstlisting}
bw_grams.items()
\end{lstlisting}
\item List keys
\begin{lstlisting}
bw_grams.keys()
\end{lstlisting}
\item List values
\begin{lstlisting}
bw_grams.values()
\end{lstlisting}
\item Iterable
\begin{lstlisting}
tmp= bw_grams.iteritems()
for i in tmp: 
  print i
\end{lstlisting}
\item Compare
\begin{lstlisting}
for i in bw_grams: 
  print j
\end{lstlisting}
\end{itemize}
\end{frame}

\end{document}