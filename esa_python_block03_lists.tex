\documentclass{beamer}

\usetheme{default}
\beamertemplatenavigationsymbolsempty

%http://mikedewar.wordpress.com/2009/02/25/latex-beamer-python-beauty/
\definecolor{fore}{RGB}{249,242,215}
%{249,242,215}
\definecolor{back}{RGB}{51,51,51}
%{RGB}{51,51,51}
\definecolor{title}{RGB}{230,96,6}
%{255,0,90}
%http://meyerweb.com/eric/tools/color-blend/
%http://www.census.gov/population/international/data/worldpop/table_population.php
\setbeamercolor{titlelike}{fg=title}
\setbeamercolor{normal text}{fg=fore,bg=back}

\useinnertheme{rectangles}
\definecolor{UniBlue}{RGB}{34,139,34}
\setbeamercolor*{item}{fg=UniBlue}

\usepackage{colortbl}
\usepackage{caption}
\usepackage{listings,bera}
\usepackage{lipsum}
\newcommand\Fontvi{\fontsize{6}{7.2}\selectfont}
\newcommand\Fontvii{\fontsize{7}{7.5}\selectfont}
\newcommand\Fontviii{\fontsize{8}{7.8}\selectfont}
\newcommand\Fontix{\fontsize{9}{8.3}\selectfont}
\definecolor{keywords}{RGB}{255,102,0}
%{255,0,90}
\definecolor{comments}{RGB}{51,153,204}
%\definecolor{comments}{RGB}{83,121,170}
\setbeamertemplate{caption}[numbered]
%\useoutertheme{infolines}
\lstset{language=Python,
keywordstyle=\color{keywords},
commentstyle=\color{comments}\emph}

%\logo{epa-logo-full.jpg}

%example table
%\begin{frame}[fragile]
%\frametitle{Basic math operations}
%\begin{center}
%\begin{tabular}{lc} \hline
%\rowcolor{UniBlue!100}col1 & col2 \\ \hline \hline
%\rowcolor{UniBlue!75}3 & 3 \\ \hline
%\rowcolor{UniBlue!90}3 & 3 \\ \hline
%\rowcolor{UniBlue!75}3 & 3 \\ \hline
%\rowcolor{UniBlue!90}3 & 3 \\ \hline
%\end{tabular}
%\end{center}

% items enclosed in square brackets are optional; explanation below
\title[Title1]{Lists}
\subtitle[Title2]{Python for Ecologists}
\author[etal]{Jon Flaishans, Tom Purucker, Tao Hong, Marcia Snyder}
\institute[EPA]{
  Ecological Society of America Workshop\\
  Minneapolis, MN\\[1ex]
  \texttt{purucker.tom@gmail.com}
}

\begin{document}

%--- the titlepage frame -------------------------%
\begin{frame}[plain]
  \titlepage
\end{frame}

%\begin{frame}
%\frametitle{Frame with reduced font size}
%\Fontvi
%\lipsum[1]
%\end{frame}

%\begin{frame}
%\frametitle{Frame with regular font size}
%\lipsum[1]
%\end{frame}

%\begin{frame}[fragile]
%\frametitle{Generic slide}
%\begin{itemize}
%\item thing 1  
%\item thing 2 
%\item thing 3 
%\item thing 4
%\end{itemize} 
%\end{frame}


%example table
\begin{frame}[fragile]
\frametitle{Lists, Tuples, and Dictionaries}
\begin{center}
\begin{tabular}{lccc} \hline
\rowcolor{UniBlue!100}Type & Create Empty & Mutable? & Order   \\ \hline \hline
\rowcolor{UniBlue!75}List & my\_list = [] & Mutable & Yes \\ \hline
\rowcolor{UniBlue!90}Tuple & my\_tuple = () & Immutable & No \\ \hline
\rowcolor{UniBlue!75}Dictionary & my\_dictionary = \{\} & Mutable & No \\ \hline
\end{tabular}
\end{center}
\begin{itemize}
\item Mutable here means you can append, change, subtract, etc.
\end{itemize}
\end{frame}

\begin{frame}[fragile]
\frametitle{Lists}
\begin{lstlisting}
species_names = []

species_names.append("Geospiza fuliginosa") # small

species_names.extend(
  ['Geospiza fortis','Geospiza magnirostris']
  ) # medium, large
  
sorted(species_names) #produces copy of list, sorted

species_names.sort() #sorts existing list (mutable)
\end{lstlisting}
\end{frame}

\begin{frame}[fragile]
\frametitle{Lists}
\begin{itemize}
\item Index, Value pairs
\item Lists can be nested
\item Slices, Length, and Index
\item Tuples can use any immutable type as an index \\
  (not just integers).....more on this later
\end{itemize}
\end{frame}

\begin{frame}[fragile]
\frametitle{Some list functions}
\begin{itemize}
\item Lists can mix types
\begin{lstlisting}
some_list = [23, 23., 'Frog', None, True] 
#None and Boolean types
\end{lstlisting}
\item Lists have similar methods as strings
\begin{lstlisting}
some_list[0]
len(some_list)
[1,2] + [3,4]
\end{lstlisting}
\item We can easily loop over the list elements
\begin{lstlisting}
for thing in some_list: 
  print thing
\end{lstlisting}
\item And check to see if elements are in the list
\begin{lstlisting}
'Frog' in some_list
'Bird' in some_list
\end{lstlisting}
\end{itemize}
\end{frame}

\begin{frame}[fragile]
\frametitle{Deleting list elements}
\begin{itemize}
\item Getting rid of list elements:
\begin{lstlisting}
some_list = ["list item1","list item2","list item3"]

some_list.pop(0)
  # removes last item from list & returns its value

some_list.insert(0,"list item1") # re-add the item

del some_list[2] # must be int (index value)

del some_list

some_list.remove("list item1") 
  # removes only 1st occurance in list
\end{lstlisting}
\end{itemize}
\end{frame}

\begin{frame}[fragile]
\frametitle{Tuples}
\begin{itemize}
\item Tuples are immutable objects that cannot be altered
\item Use parentheses () instead of square brackets []
\begin{lstlisting}
some_tuple = [
  ("Frog1",23.2),
  ("Frog1",20.8),
  ]
\end{lstlisting}
\item Tuples and lists can both be sliced
\begin{lstlisting}
some_tuple[0:2]
some_list[0:2]
\end{lstlisting}
\item ....and sorted
\begin{lstlisting}
sorted(some_tuple, key=lambda grams:grams[1])
\end{lstlisting}
\end{itemize}
\end{frame}

\begin{frame}[fragile]
\frametitle{Exercise 3- Run the script exer03\_lists.py}
\Fontvi
\begin{lstlisting}
class TestLists(unittest.TestCase):
    def test_lists(self):
        """
        A basic introduction to lists
        """
        # Create the variable ``bird_list`` and assign to an empty list
        # **************************************************

        self.assertEquals(bird_list, [])

        # Append 'American redstart' and 'Arctic tern' to ``bird_list``
        # **************************************************

        self.assertEquals(bird_list, ['American redstart','Arctic tern' ])

        # Sort ``bird_list``
        # **************************************************

        self.assertEquals(bird_list,  ['Arctic tern', 'American redstart'])

        # ``extend`` the list ``bird_list`` with ['Northern parula', 'george']
        # **************************************************

        self.assertEquals(bird_list, ['Arctic tern', 'American redstart', ...])

        # create a variable ``warbler_id`` with the index of 'Hooded warbler' in
        # ``bird_list`` using list methods.
        # **************************************************

        self.assertEquals(warbler_id, 3)
\end{lstlisting}
\end{frame}


\end{document}